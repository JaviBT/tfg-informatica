% Autor: Francisco Javier Barranco Tena
% Introducción
% Alt + z o Option + z para activar el word wrap en Visual Studio Code

La problemática de los accidentes de tráfico representa una preocupación a nivel global. Cada año, aproximadamente 1.35 millones de personas pierden la vida en las carreteras de todo el mundo, con casi 3,700 muertes diarias en accidentes que involucran una variedad de vehículos y peatones\cite{WHO_Road_Safety}. Estas cifras alarmantes sitúan las lesiones por accidentes de tráfico como la octava causa de muerte a nivel mundial, siendo la principal causa de fallecimiento para personas de entre 5 y 29 años, superando incluso a enfermedades como el VIH/SIDA\cite{WHO_Road_Safety}. Además del costo humano, estas tragedias representan una carga económica considerable, estimada en aproximadamente \$1.8 billones de dólares para la economía mundial entre 2015 y 2030, equivalente a un impuesto anual del 0.12\% sobre el PIB mundial\cite{MACROECON_Road_Safety}.

Es importante destacar que esta problemática afecta de manera desproporcionada a los países de bajos y medios ingresos LMICs\footnote{Low- and Middle-Income Countries (LMICs) son países de ingresos bajos y medianos, según la clasificación del Banco Mundial, que abarcan una amplia gama de economías en desarrollo. Estos países se caracterizan por tener ingresos per cápita más bajos en comparación con los países de ingresos altos, y a menudo enfrentan desafíos significativos en áreas como la salud, la educación y la infraestructura.}, donde la tasa de mortalidad por accidentes de tráfico es más de tres veces superior a la de los países de altos ingresos\cite{WHO_Road_Safety}. Aunque representan solo el 60\% de los vehículos registrados a nivel mundial, los países de bajos y medianos ingresos soportan más del 90\% de las muertes por accidentes de tráfico. Esta situación no solo tiene un impacto humano devastador, sino que también impone una carga económica significativa a estos países, estimada en unos \$834 mil millones de dólares en pérdidas económicas entre 2015 y 2030 debido a lesiones fatales y no fatales causadas por accidentes de tráfico\cite{MACROECON_Road_Safety}.

El estado del pavimento juega un papel crucial en la seguridad vial, siendo un factor determinante para prevenir accidentes. La falta de mantenimiento del pavimento no solo compromete la seguridad, sino que también da lugar a una serie de problemas adicionales, como la ralentización del tráfico, el aumento de atascos y los daños a los vehículos. Es esencial abordar esta problemática de manera efectiva, y es por ello que este trabajo se enfocará en el estudio y aplicación de nuevas tecnologías, específicamente la inteligencia artificial, para detectar y evaluar el estado del pavimento.

La utilización de inteligencia artificial en la detección de imperfecciones en el pavimento permite reducir los costos asociados al mantenimiento, ya que aumenta la eficiencia del proceso de inspección y detección de daños. Además, aumenta la capacidad de anticipación a posibles daños, lo que permite una respuesta más temprana y, por tanto, reduce los potenciales daños causados por un mal estado de las vías. Todo esto hace que sistemas basados en inteligencia artificial sean una herramienta valiosa para gobiernos locales y empresas encargadas del mantenimiento de las vías.

En este escenario, el uso de tecnologías como \textit{Computer Vision}\footnote{Computer Vision es un anglicismo que se traduce al español como Visión por Computador.} ofrece una solución innovadora para evaluar y prever el estado de las vías de manera más eficiente y asequible. Este enfoque no solo optimiza los costos asociados al mantenimiento, sino que también agiliza el proceso de detección de imperfecciones, permitiendo una respuesta más rápida y eficaz ante la necesidad de reparaciones.

Este trabajo de fin de grado se enfoca en explorar, aplicar y evaluar el estado actual de las tecnologías de \textit{Computer Vision} que pueden ser empleadas para la detección del estado de pavimentos. El objetivo principal es implementar un sistema base que permita la detección de estado de pavimento mediante \textit{Computer Vision}. Dicho objetivo principal se desglosa en los siguientes objetivos específicos: estudio del estado del arte o SoA\footnote{State of the Art, en inglés, se refiere al estado del arte, es decir, al nivel de desarrollo alcanzado en un campo de conocimiento en un momento determinado.}, selección del marco de evaluación, diseño e implementación de sistema base, y evaluación de resultados. Estos objetivos se abordarán a lo largo de los capítulos de este trabajo, que se estructuran de la siguiente manera:

\begin{itemize}
    \item \textbf{Capítulo 2: Estado del arte.} En este capítulo se presenta una revisión de la literatura existente en el campo de la detección de estado de pavimento mediante \textit{Computer Vision}. También se describe el modelo de detección de objetos YOLOv8, que se utilizará para implementar el sistema base.
    
    \item \textbf{Capítulo 3: Diseño y desarrollo.} En este capítulo se detalla el marco de evaluación seleccionado para evaluar el sistema base implementado. Luego, se describe la metodología seguida para entrenar, validar e inferir con el modelo YOLOv8.
    
    \item \textbf{Capítulo 4: Experimentos y resultados.} En este capítulo se presentan los experimentos iterativos que han culminado en el sistema base implementado. Se detallan los resultados obtenidos en cada iteración y se analizan los resultados finales del sistema base.
    
    \item \textbf{Capítulo 5: Conclusiones y trabajo futuro.} En este capítulo se presentan las conclusiones obtenidas a partir de los resultados obtenidos en los experimentos. Se discuten las limitaciones del sistema base implementado y se proponen posibles líneas de trabajo futuro para mejorar y ampliar el alcance de este trabajo.
\end{itemize}
