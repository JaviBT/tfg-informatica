% Autor: Francisco Javier Barranco Tena
% Introducción
% Alt + z o Option + z para activar el word wrap en Visual Studio Code

El estado de las carreteras y aceras juega un papel crucial en la seguridad vial, siendo un factor determinante para prevenir accidentes. La falta de mantenimiento del pavimento no solo compromete la seguridad, sino que también da lugar a una serie de problemas adicionales, como la ralentización del tráfico, el aumento de atascos y los daños a los vehículos. Es esencial abordar esta problemática de manera efectiva, y es por ello que este trabajo se enfocará en la investigación y aplicación de nuevas tecnologías, específicamente la inteligencia artificial, para detectar y evaluar el estado del pavimento y las aceras.

La utilización de inteligencia artificial en la detección de imperfecciones en el pavimento y las aceras permite reducir los costos asociados al mantenimiento, ya que aumenta la eficiencia del proceso de inspección y detección de daños. Además, aumenta la capacidad de anticipación a posibles daños, lo que permite una respuesta más temprana y, por tanto, reduce los potenciales daños causados por un mal estado de las vías. Todo esto hace que sistemas basados en inteligencia artificial sean una herramienta valiosa para gobiernos locales y empresas encargadas del mantenimiento de las vías.

En este escenario, el uso de tecnologías como Computer Vision ofrece una solución innovadora para evaluar y prever el estado de las vías de manera más eficiente y asequible. Este enfoque no solo optimiza los costos asociados al mantenimiento, sino que también agiliza el proceso de detección de imperfecciones, permitiendo una respuesta más rápida y eficaz ante la necesidad de reparaciones.

Este trabajo de fin de grado se enfoca en explorar el estado actual de las tecnologías de Computer Vision que pueden ser empleadas para la detección del estado de pavimentos y aceras. El objetivo principal es evaluar la viabilidad de estas tecnologías para la detección de imperfecciones en las vías, así como evaluar la posibilidad de implementar un sistema eficaz y de bajo coste que permita una evaluación precisa y temprana del estado de las vías.