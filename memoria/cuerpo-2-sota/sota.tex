% Autor: Francisco Javier Barranco Tena
% State of the Art
% Alt + z o Option + z para activar el word wrap en Visual Studio Code

Detectar el estado del pavimento y las aceras presenta desafíos significativos que han sido abordados mediante diversas soluciones. Entre estas soluciones se incluyen la integración de sensores de acelerómetros en vehículos [paper], la aplicación de sistemas basados en LIDAR [paper] , y el desarrollo de sistemas fundamentados en Computer Vision. Los avances recientes en algoritmos de detección de objetos, como YOLO (You Only Look Once), han mejorado notablemente la eficacia de los sistemas basados en Visión por Computadora. Además, estos sistemas no requieren una inversión significativa en hardware específico, ya que pueden emplearse con cámaras convencionales, como las de teléfonos móviles o las cámaras de vehículos.

En el ámbito de soluciones fundamentadas en Computer Vision, destaca la CRDDC (Crowdsensing-based Road Damage Detection Challenge), una competición que ha tenido lugar en los años 2018, 2020 y 2022. El propósito de la CRDDC radica en la implementación de sistemas de Computer Vision capaces de utilizar cámaras de teléfonos móviles y vehículos para inspeccionar el estado del pavimento en cualquier parte del mundo. Esta competición proporciona un extenso conjunto de datos compuesto por decenas de miles de imágenes anotadas, así como un método de evaluación que facilita la comparación de los modelos desarrollados con aquellos implementados durante la competición.

En el artículo “Crowdsensing-based Road Damage Detection Challenge (CRDDC’2022)” \cite{CRDDC2022_paper}, se explica de forma resumida las soluciones propuestas por los once equipos líderes de la competición CRDDC 2022. Dentro de esta lista de soluciones propuesta, destaca que casi todas las soluciones utilizan ‘ensembles’ basados principalmente en modelos YOLO (You Only Look Once). Adicionalmente, el artículo "From global challenges to local solutions: A review of cross-country collaborations and winning strategies in road damage detection." \cite{CRDDC2022_review} proporciona un resumen de las soluciones y aprendizajes de la edición de 2022 de la competición. Durante este TFG utilizaremos los datos, soluciones propuestas, aprendizajes y métricas de la CRDDC 2022 para desarrollar un modelo de detección de estado del pavimento y aceras.

Como se ha mencionado, en la CRDDC 2022 destaco el uso de los modelos YOLO. YOLO \cite{YOLO} es un modelo de detección de objetos en tiempo real que divide la imagen en una cuadrícula y predice las cajas delimitadoras y las clases de los objetos en cada celda de la cuadrícula. En este TFG se utilizará la implementación de YOLOv8 de Ultralytics \cite{yolov8_ultralytics}, que es una de las implementaciones más avanzadas y eficientes de YOLO. La librería de Python de Ultralytics proporciona una interfaz sencilla para entrenar, validar y predecir con modelos YOLO.

Generalmente, en competiciones como la CRDDC, los modelos ganadores suelen ser ensembles complejos que combinan múltiples modelos de detección de objetos. Estos ensembles suelen combinar modelos de diferentes arquitecturas, como YOLO, Faster R-CNN, o EfficientDet, y se entrenan utilizando técnicas de transfer learning. En este trabajo buscaremos implementar un modelo más simple que no requiera tantas capacidades computacionales, pero que sea capaz de predecir daños en el pavimento con una buena precisión y eficacia en tiempo real. La idea es tener un modelo preciso, que generalice bien y que sea rápido, para poder ser implementado en un dispositivo móvil o en un vehículo.
