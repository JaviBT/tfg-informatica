En esta sección se va a explicar cómo se ha entrenado el modelo de detección de estado del pavimento. Comenzaremos explicando los parámetros de entrenamiento y otras consideraciones que se deben tener en cuenta a la hora de entrenar un modelo de detección de objetos. Después, se explicará cómo se han preparado los datos de la CRDDC2022 para tener en cuenta estas consideraciones y adaptar los datos a las particularidades de la librería Ultralytics. Por último, se explicará los requerimientos de hardware necesarios para entrenar los modelos de detección de objetos y cómo se ha llevado a cabo el entrenamiento en Google Colab y en VPULab durante el desarrollo de este TFG.

\subsection{Parámetros de entrenamiento y otras consideraciones}
Siempre que se entrena un modelo de aprendizaje profundo, como los YOLO, buscamos una serie de cosas:
\begin{itemize}
    \item Que el modelo tenga una buena capacidad de generalización y no este sobreajustado al conjunto de entrenamiento.
    \item Que el modelo tenga altas puntuaciones de precisión y recall.
    \item Que el entrenamiento sea rápido y eficiente.
    \item \dots
    \item \dots
\end{itemize}
Para conseguir estos objetivos se deben tener en cuenta múltiples factores a la hora de preparar los datos y conjuntos de entrenamiento, validación y prueba; elegir el modelo; y configurar los parámetros de entrenamiento. En esta subsección se va a explicar cómo se han configurado los parámetros de entrenamiento y se van a dar algunas recomendaciones para entrenar modelos de detección de objetos.

% TODO: Añadir más información sobre los parámetros de entrenamiento y otras consideraciones
...

\subsection{Preparación de los datos}
En la sección de modelos [Sección \ref{SEC:MODELOS}] se ha explicado cómo se ha usado YOLO de Ultralytics para la detección del estado del pavimento. La elección de dicha librería nos obliga a realizar una serie de transformaciones en los datos para poder entrenar nuestros modelos. En esta subsección se va a explicar cómo se han preparado los datos de la CRDDC2022 para poder ser usados con YOLO de Ultralytics. Estos cambios se pueden encontrar detallados en los notebooks 'prepare-DatasetNinja.ipynb' y 'dataToYoloFormat.ipynb' en el repositorio del TFG \cite{TFG_Repository}.

% TODO: Añadir más información sobre la preparación de los datos
...

\subsection{Entrenamiento en Google Colbas}
resumen

\subsection{Entrenamiento en VPULab}
resumen