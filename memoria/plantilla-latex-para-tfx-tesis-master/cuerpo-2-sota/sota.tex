Detectar el estado del pavimento y las aceras presenta desafíos significativos que han sido
abordados mediante diversas soluciones. Entre estas soluciones se incluyen la integración de
sensores de acelerómetros en vehículos [paper], la aplicación de sistemas basados en LIDAR
[paper] , y el desarrollo de sistemas fundamentados en Computer Vision. Los avances
recientes en algoritmos de detección de objetos, como YOLO (You Only Look Once), han
mejorado notablemente la eficacia de los sistemas basados en Visión por Computadora.
Además, estos sistemas no requieren una inversión significativa en hardware específico, ya
que pueden emplearse con cámaras convencionales, como las de teléfonos móviles o las
cámaras de vehículos.

En el ámbito de soluciones fundamentadas en Computer Vision, destaca la CRDDC
(Crowdsensing-based Road Damage Detection Challenge), una competición que ha tenido
lugar en los años 2018, 2020 y 2022. El propósito de la CRDDC radica en la implementación
de sistemas de Computer Vision capaces de utilizar cámaras de teléfonos móviles y vehículos
para inspeccionar el estado del pavimento en cualquier parte del mundo. Esta competición
proporciona un extenso conjunto de datos compuesto por decenas de miles de imágenes
anotadas, así como un método de evaluación que facilita la comparación de los modelos
desarrollados con aquellos implementados durante la competición.

En el artículo “Crowdsensing-based Road Damage Detection Challenge (CRDDC’2022)”
[paper], se explica de forma resumida las soluciones propuestas por los once equipos líderes
de la competición CRDDC 2022. Dentro de esta lista de soluciones propuesta, destaca que
casi todas las soluciones utilizan ‘ensembles’ basados principalmente en modelos YOLO (You
Only Look Once). Adicionalmente, el paper From global challenges ... [paper] proporciona un resumen de las soluciones y aprendizajes de la edición de 2022 de la competición. Durante este TFG utilizaremos los datos, soluciones propuestas, aprendizajes y métricas de la CRDDC 2022 para desarrollar un modelo de detección de estado del pavimento y aceras.

[AQUÍ SE PUEDE EXPLICAR QUE ES UN ENSEMBLE/AGREGACION DE MODELOS]
[SI HAY ESPACIO, SE PODRIA AÑADIR UNA EXPLICACION MAS DETALLADA DE YOLO Y COMO
FUNCIONAN LAS CNN]

Como se ha mencionado, en la CRDDC 2022 destaco el uso de los modelos YOLO. YOLO [cite] es un 
SERIA APROPIADA CITAR PAPERS, ARTICULOS, ETC. QUE POSICIONEN A YOLO COMO UNO DE LOS MEJORES MODELOS EN ESTE AMBITO

[TENEMOS QUE HABLAR DE ULTRALYTICS CON EXPLICACION Y CITA]

El objetivo inicial de este proyecto consistirá en la implementación de una de las soluciones
destacadas durante la CRDDC, con una preferencia por aquellas presentadas en la edición de
2022. Posteriormente, se llevará a cabo un proceso iterativo de mejora del modelo,
introduciendo ajustes y optimizaciones. La evaluación del rendimiento se llevará a cabo
tanto en la plataforma de la competición como de manera local, utilizando métricas
similares. La meta final es la creación de un modelo que no solo supere el estado del arte
actual en la detección del estado del pavimento, sino que también introduzca innovaciones
significativas y de interés en este campo.
[AÑADIR DETALLES DEL MODELO BASE QUE IMPLEMENTE]