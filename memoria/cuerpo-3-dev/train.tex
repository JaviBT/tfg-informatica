En esta sección se va a explicar cómo se ha entrenado el modelo de detección de estado del pavimento. Comenzaremos explicando los parámetros de entrenamiento y otras consideraciones que se deben tener en cuenta a la hora de entrenar un modelo de detección de objetos. Después, se explicará cómo se han preparado los datos de la CRDDC2022 para tener en cuenta estas consideraciones y adaptar los datos a las particularidades de la librería Ultralytics. Por último, se explicará los requerimientos de hardware necesarios para entrenar los modelos de detección de objetos y cómo se ha llevado a cabo el entrenamiento en Google Colab y en VPULab durante el desarrollo de este TFG.

\subsection{Parámetros de entrenamiento y otras consideraciones}
Siempre que se entrena un modelo de aprendizaje profundo, como los YOLO, buscamos una serie de cosas:
\begin{itemize}
    \item Que el modelo tenga una buena capacidad de generalización y no este sobreajustado al conjunto de entrenamiento.
    \item Que el modelo tenga altas puntuaciones de precisión y recall.
    \item Que el entrenamiento sea rápido y eficiente.
    \item \dots
    \item \dots
\end{itemize}
Para conseguir estos objetivos se deben tener en cuenta múltiples factores a la hora de preparar los datos y conjuntos de entrenamiento, validación y prueba; elegir el modelo; y configurar los parámetros de entrenamiento. En esta subsección se va a explicar cómo se han configurado los parámetros de entrenamiento y se van a dar algunas recomendaciones para entrenar modelos de detección de objetos.

En el caso de los modelos YOLO, se debe tener en cuenta el tamaño de batch, el numero de épocas, el tamaño del modelo que se va a utilizar, la proporción de los datos anotados que se usa para entrenamiento y validación. .........

\subsection{Preparación de los datos}
En la sección de modelos [Sección \ref{SEC:MODELOS}] se ha explicado cómo se ha usado YOLO de Ultralytics para la detección del estado del pavimento. La elección de dicha librería nos obliga a realizar una serie de transformaciones en los datos para poder entrenar nuestros modelos. En esta subsección se va a explicar cómo se han preparado los datos de la CRDDC2022 para poder ser usados con YOLO de Ultralytics. Estos cambios se pueden encontrar detallados en los notebooks 'prepare-DatasetNinja.ipynb' y 'dataToYoloFormat.ipynb' en el repositorio del TFG \cite{TFG_Repository}.

Ultralytics requiere que las anotaciones tengan un formato especifico para poder ser usadas en el entrenamiento de los modelos. En concreto, las anotaciones deben estar en un directorio llamado 'labels' que esté en el mismo directorio que el directorio 'images' que contiene las imágenes. Cada archivo de anotación debe tener el mismo nombre que el archivo de imagen correspondiente, pero con extensión '.txt' en lugar de '.jpg'. El contenido de cada archivo de anotación debe tener una línea por cada objeto en la imagen con el siguiente formato:
\begin{center}
    \texttt{<clase> <x> <y> <width> <height>}
\end{center}
Donde \texttt{<clase>} es el índice de la clase del objeto, \texttt{<x>} y \texttt{<y>} son las coordenadas del centro de la caja delimitadora normalizadas, y \texttt{<width>} y \texttt{<height>} son el ancho y el alto de la caja delimitadora normalizados. Las coordenadas normalizadas se calculan dividiendo las coordenadas de la caja delimitadora por el ancho y el alto de la imagen, respectivamente. Esta transformación se han realizado con el notebook 'prepare-DatasetNinja.ipynb' en el repositorio del TFG \cite{TFG_Repository}. Se han realizado otros cambios menos significativos como reducir el tamaño de las imágenes de Noruega para que sea mas manejable en Google Colab.

\subsection{Entrenamiento en Google Colabs}
resumen .......

\subsection{Entrenamiento en VPULab}
resumen .......