Los modelos YOLO (You Only Look Once) son ampliamente utilizados en la detección de objetos en imágenes. En esta sección se explicará cómo se ha usado YOLO de Ultralytics, una de las librerías más utilizadas para la implementación de modelos de detección de objetos en imágenes. Además, se explicará cómo se ha adaptado el modelo YOLO para la detección del estado del pavimento.

\subsection{YOLO en Ultralytics}

Ultralytics es una librería de código abierto que proporciona una implementación de múltiples versiones de los modelos YOLO. En concreto, Ultralytics proporciona la versión YOLOv8 \cite{yolov8_ultralytics}, que es una versión mejorada de YOLOv5 con mejoras en la velocidad y la precisión. La librería de Ultralytics proporciona una interfaz sencilla para entrenar, evaluar y hacer inferencias con los modelos YOLO.

Para entrenar un modelo YOLO con Ultralytics, primero se necesita un conjunto de datos con imágenes y sus correspondientes anotaciones. Estas anotaciones deben estar en formato COCO, que es un formato estándar para anotaciones de objetos en imágenes. Una vez se tiene el conjunto de datos, se puede entrenar un modelo YOLO con Ultralytics con un solo comando. Ultralytics proporciona una serie de parámetros que se pueden ajustar para personalizar el entrenamiento del modelo, como el tamaño de las imágenes, el número de clases, el número de épocas, etc. Además, Ultralytics proporciona una serie de métricas de evaluación que se pueden utilizar para evaluar el rendimiento del modelo. Estas métricas incluyen la precisión, el recall, el F1-score, etc. y se veran en detalle en la sección de métricas de este capítulo [Sección \ref{SEC:METRICAS}].
